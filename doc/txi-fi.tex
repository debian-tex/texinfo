% txi-fi.tex -- adaptation to Finnish for texinfo.tex.
%
% Copyright (C) 2020 Free Software Foundation
% 
% This program is free software: you can redistribute it and/or modify
% it under the terms of the GNU General Public License as published by
% the Free Software Foundation, either version 3 of the License, or
% (at your option) any later version.
% 
% This program is distributed in the hope that it will be useful,
% but WITHOUT ANY WARRANTY; without even the implied warranty of
% MERCHANTABILITY or FITNESS FOR A PARTICULAR PURPOSE.  See the
% GNU General Public License for more details.
% 
% You should have received a copy of the GNU General Public License
% along with this program.  If not, see <https://www.gnu.org/licenses/>.
%
% Finnish was written by Juha-Matti Huusko, 30.12.2019, juha-matti.huusko@uef.fi
%%
\txisetlanguage{finnish}{2}{2}
%
%\gdef\putwordNext{Seuraava} %ei v�ltt�m�tt� toimi
%\gdef\putwordPrevious{Edellinen}
\gdef\putwordAppendix{Liitteet}
\gdef\putwordChapter{Kirja}
\gdef\putwordfile{tiedosto}
\gdef\putwordin{in}
\gdef\putwordInfo{Info}
\gdef\putwordMethodon{Methode von}
\gdef\putwordon{, teki, }%auf
\gdef\putwordof{, paikasta, }%von
\gdef\putwordpage{sivu}
\gdef\putwordsection{luku}
\gdef\putwordSection{Luku}
\gdef\putwordsee{katso}
\gdef\putwordSee{Katso}
\gdef\putwordShortTOC{Lyhyt sis�llysluettelo}
\gdef\putwordTOC{Sis�llysluettelo}
%%
\gdef\putwordNoTitle{Ei otsikkoa}
%%
%% New defintion for the output of months.
\gdef\putwordMJan{Tammikuu}
\gdef\putwordMFeb{Helmikuu}
\gdef\putwordMMar{Maaliskuu}
\gdef\putwordMApr{Huhtikuu}
\gdef\putwordMMai{Toukokuu}
\gdef\putwordMJun{Kes�kuu}
\gdef\putwordMJul{Hein�kuu}
\gdef\putwordMAug{Elokuu}
\gdef\putwordMSep{Syyskuu}
\gdef\putwordMOct{Lokakuu}
\gdef\putwordMNov{Marraskuu}
\gdef\putwordMDec{Joulukuu}
%%
\gdef\putwordIndexNonexistent{(Indexi� ei ole olemassa)}
\gdef\putwordIndexIsEmpty{(Index on tyhj�)}
%%
%% \defmac
\gdef\putwordDefmac{Makro}
%% \defspec
\gdef\putwordDefspec{Spezial Form}
%% \defivar
\gdef\putwordDefivar{esimerkkimuuttuja}
%% \defvar
\gdef\putwordDefvar{Variable}
%% \defopt
\gdef\putwordDefopt{k�ytt�j�vaihtoehto}
%% \deftypevar
\gdef\putwordDeftypevar{Muuttuja}
%% \deffun
\gdef\putwordDeffunc{Funktio}
%% \deftypefun
\gdef\putwordDeftypefun{funktio}
